% SIAM Shared Information Template
% This is information that is shared between the main document and any
% supplement. If no supplement is required, then this information can
% be included directly in the main document.


% Packages and macros go here
\usepackage{lipsum}
\usepackage{amsfonts}
\usepackage{graphicx}
\usepackage{epstopdf}
\usepackage{algorithmic}
\ifpdf
  \DeclareGraphicsExtensions{.eps,.pdf,.png,.jpg}
\else
  \DeclareGraphicsExtensions{.eps}
\fi

% Prevent itemized lists from running into the left margin inside theorems and proofs
\usepackage{enumitem}
\setlist[enumerate]{leftmargin=.5in}
\setlist[itemize]{leftmargin=.5in}

% Add a serial/Oxford comma by default.
\newcommand{\creflastconjunction}{, and~}

% Used for creating new theorem and remark environments
\newsiamremark{remark}{Remark}
\newsiamremark{hypothesis}{Hypothesis}
\crefname{hypothesis}{Hypothesis}{Hypotheses}
\newsiamthm{claim}{Claim}
\newsiamremark{fact}{Fact}
\crefname{fact}{Fact}{Facts}

% Sets running headers as well as PDF title and authors
\headers{A Unified Analysis of Generalization and Sample Complexity for Semi-Supervised Domain Adaptation}{Elif Vural, Huseyin Karaca}

% Title. If the supplement option is on, then "Supplementary Material"
% is automatically inserted before the title.
\title{A Unified Analysis of Generalization and Sample Complexity for Semi-Supervised Domain Adaptation\thanks{Submitted to the editors \today.
\funding{This work is supported by The Scientific and Technological Research Council of Türkiye (TÜBİTAK) 1515 Frontier R\&D Laboratories Support Program for Türk Telekom 6G R\&D Lab under project number 5249902 and 2210 National Graduate Scholarship Program.}}}

% Authors: full names plus addresses.
\author{Elif Vural\thanks{Department of Electrical and Electronics Engineering, METU, Ankara, Türkiye 
  (\email{velif@metu.edu.tr}, \url{http://blog.metu.edu.tr/velif/}).}
\and Hüseyin Karaca\thanks{Department of Electrical and Electronics Engineering, Bilkent University, Ankara, Türkiye and Türk Telekom, Ankara, Türkiye 
  (\email{huseyin.karaca@bilkent.edu.tr}, \email{hkaraca@turktelekom.com.tr}, \url{https://huseyin-karaca.github.io}).}}

\usepackage{amsopn}
\DeclareMathOperator{\diag}{diag}


% Commands for symbols
% General
\newcommand{\half}{\ensuremath{ \frac{1}{2} }}
\newcommand{\tr}{\ensuremath{ \mathrm{tr}}}
\newcommand{\R}{\ensuremath{\mathbb{R}}}
\newcommand{\M}{\ensuremath{\mathcal{M}}}
\newcommand{\Rn}{\ensuremath{\mathbb{R}^n}}
\newcommand{\E}{\ensuremath{\mathbb{E}}}
\newcommand{\B}{\ensuremath{\mathcal{B}}}
\newcommand{\wvect}{\ensuremath{\mathbf{w}}}
% Domain 
\newcommand{\Xs}{\ensuremath{\mathcal{X}^s}} % Source data set
\newcommand{\Xt}{\ensuremath{\mathcal{X}^t}} %Target data set
\newcommand{\Ss}{\ensuremath{X^s}} % Source sample set
\newcommand{\St}{\ensuremath{X^t}} %Target sample set
\newcommand{\Y}{\ensuremath{\mathcal{Y}}} % Label set
\newcommand{\Zs}{\ensuremath{\mathcal{Z}^s}} % Source combined set
\newcommand{\Zt}{\ensuremath{\mathcal{Z}^t}} %Target combined set
\newcommand{\mus}{\ensuremath{\mu_s}} % Source probability measure
\newcommand{\mut}{\ensuremath{\mu_t}} % Target probability measure
\newcommand{\nus}{\ensuremath{\nu_s}} % Source probability measure
\newcommand{\nut}{\ensuremath{\nu_t}} % Target probability measure

% Variance and moments
\newcommand{\vars}{\ensuremath{\sigma_s^2}}
\newcommand{\vart}{\ensuremath{\sigma_t^2}}
\newcommand{\Cs}{\ensuremath{C_s}}
\newcommand{\Ct}{\ensuremath{C_t}}

% Data
\newcommand{\xs}{\ensuremath{x^s}} % Source data sample
\newcommand{\xt}{\ensuremath{x^t}} %Target data sample
\newcommand{\xis}{\ensuremath{x^s_i}} % Source data sample
\newcommand{\xjs}{\ensuremath{x^s_j}} % Source data sample
\newcommand{\xjt}{\ensuremath{x^t_j}} %Target data sample
\newcommand{\xit}{\ensuremath{x^t_i}} %Target data sample
\newcommand{\ys}{\ensuremath{\mathbf{y}^s}} % Source label
\newcommand{\yt}{\ensuremath{\mathbf{y}^t}} %Target label
\newcommand{\yis}{\ensuremath{\mathbf{y}^s_i}} % Source label
\newcommand{\y}{\ensuremath{\mathbf{y}}} %Target label
\newcommand{\yjt}{\ensuremath{\mathbf{y}^t_j}} %Target label
\newcommand{\zs}{\ensuremath{z^s}} % Source sample
\newcommand{\zt}{\ensuremath{z^t}} %Target sample
%
\newcommand{\Ns}{\ensuremath{N_s}} %Number of source data samples
\newcommand{\Nt}{\ensuremath{N_t}} %Number of target data samples
%
\newcommand{\Ms}{\ensuremath{M_s}} %Number of labeled source data samples
\newcommand{\Mt}{\ensuremath{M_t}} %Number of labeled target data samples
%
\newcommand{\as}{\ensuremath{a_s}}
\newcommand{\at}{\ensuremath{a_t}}
% Mapping
\newcommand{\X}{\ensuremath{\mathcal{X}}} % Space of embedding
\newcommand{\fs}{\ensuremath{f^s}} % Source mapping
\newcommand{\ft}{\ensuremath{f^t}} %Target mapping
\newcommand{\h}{\ensuremath{h}} % Hypothesis 
\newcommand{\Fs}{\ensuremath{\mathcal{F}^s}} % Source transformation space
\newcommand{\Ft}{\ensuremath{\mathcal{F}^t}} %Target transformation space
\newcommand{\Hs}{\ensuremath{\mathcal{H}}} % Hypothesis space
\newcommand{\F}{\ensuremath{\mathcal{F}}} % Generic function space
\newcommand{\gs}{\ensuremath{g^s}} % Hypothesis
\newcommand{\gt}{\ensuremath{g^t}} % Hypothesis 
 \newcommand{\Gs}{\ensuremath{\mathcal{G}^s}} % Source transformation space
\newcommand{\Gt}{\ensuremath{\mathcal{G}^t}} %Target transformation space
 \newcommand{\Dspace}{\ensuremath{\mathcal{D}}} % Generic function space
 \newcommand{\vs}{\ensuremath{v^s}} % Composite domain discriminator function 
\newcommand{\vt}{\ensuremath{v^t}} % Composite domain discriminator function
 \newcommand{\Vs}{\ensuremath{\mathcal{V}^s}} % Composite domain discriminator function 
\newcommand{\Vt}{\ensuremath{\mathcal{V}^t}}
 
 
% Loss
\newcommand{\loss}{\ensuremath{\ell}} % Source mapping
\newcommand{\Ls}{\ensuremath{\mathcal{L}^s}} % Source expected loss
\newcommand{\Lt}{\ensuremath{\mathcal{L}^t}} % Target expected loss
\newcommand{\Lw}{\ensuremath{\mathcal{L}_{\alpha}}} % Weighed expected loss
\newcommand{\hLs}{\ensuremath{\hat{\mathcal{L}}^s}} % Source empirical loss
\newcommand{\hLt}{\ensuremath{\hat{\mathcal{L}}^t}} % Target empirical loss
\newcommand{\hLw}{\ensuremath{\hat{\mathcal{L}}_{\alpha}}} % Weighed empirical loss
% Distance
\newcommand{\D}{\ensuremath{D}} % Distribution distance
\newcommand{\Ddan}{\ensuremath{\D_\ddan}} % Distribution distance
\newcommand{\hD}{\ensuremath{\hat{D}}} % Distribution distance empirical estimate
\newcommand{\hDdan}{\ensuremath{\hat{\D}_\ddan}} % Distribution distance

% Regularity
\newcommand{\LLs}{\ensuremath{R}} % ! Modified, before: L
\newcommand{\Lls}{\ensuremath{L_\ell}} %! Modified, before: \xi
\newcommand{\Lact}{\ensuremath{L}} 
\newcommand{\bls}{\ensuremath{A_\ell}} %! Modified, before: B
\newcommand{\Lk}{\ensuremath{L_K}} % Lipschitz constant of kernel kl
\newcommand{\LLsdan}{\ensuremath{\LLs_{A}}} 


% Covering number
\newcommand{\N}{\ensuremath{\mathcal{N}}}
\newcommand{\Bx}{\ensuremath{B_{\delta}(x)}}
\newcommand{\dmetric}{\ensuremath{\mathfrak{d}}} % General metric 
\newcommand{\ds}{\ensuremath{\mathfrak{d}^s}} % Source metric
%\newcommand{\dt}{\ensuremath{d^t}} % Hypothesis 
%\newcommand{\dXs}{\ensuremath{d^s_{\mathcal{X}} }} % L-inf source distance
%\newcommand{\dXt}{\ensuremath{d^t_{\mathcal{X}} }} % L-inf target distance
\newcommand{\dt}{\ensuremath{\mathfrak{d}^t}} % Target metric 
\newcommand{\dXs}{\ensuremath{\ds_{\mathcal{X}} }} % L-inf source distance
\newcommand{\dXt}{\ensuremath{\dt_{\mathcal{X}} }} % L-inf target distance
\newcommand{\dVs}{\ensuremath{\ds_{\mathcal{V}} }} % L-inf source distance
\newcommand{\dVt}{\ensuremath{\dt_{\mathcal{V}} }} % L-inf target distance
\newcommand{\Ks}{\ensuremath{\kappa^s}} %Cover number source
\newcommand{\Kt}{\ensuremath{\kappa^t}} % Cover number target
\newcommand{\Kl}{\ensuremath{\kappa^l}} % Cover number for parameter domain of layer l
\newcommand{\Kgen}{\ensuremath{\kappa}} % General cover number symbol
\newcommand{\radTheta}{\ensuremath{\delta}} % Cover radius in parameter domain
\newcommand{\grid}{\ensuremath{\mathfrak{G}}}


% Deep DA
\newcommand{\fsl}{\ensuremath{f^{sl}}} % Source function layer l
\newcommand{\ftl}{\ensuremath{f^{tl}}} %Target function layer l
\newcommand{\fslm}{\ensuremath{f^{s(l-1)}}} % Source function layer l-1
\newcommand{\ftlm}{\ensuremath{f^{t(l-1)}}} %Target function layer l-1
\newcommand{\fsone}{\ensuremath{f^{s1}}} % Source function layer 1
\newcommand{\ftone}{\ensuremath{f^{t1}}} %Target function layer 1
\newcommand{\fsLm}{\ensuremath{f^{s(L-1)}}} % Source function layer 1
\newcommand{\ftLm}{\ensuremath{f^{t(L-1)}}} %Target function layer 1

% Hidden layers
\newcommand{\hsl}{\ensuremath{\boldsymbol{\xi}^{sl}}} % Source hidden layer l
\newcommand{\htl}{\ensuremath{\boldsymbol{\xi}^{tl}}} %Target hidden layer l
\newcommand{\hslm}{\ensuremath{\boldsymbol{\xi}^{s(l-1)}}} % Source hidden layer l-1
\newcommand{\htlm}{\ensuremath{\boldsymbol{\xi}^{t(l-1)}}} %Target hidden layer l-1
\newcommand{\hsz}{\ensuremath{\boldsymbol{\xi}^{s0}}}
\newcommand{\hsone}{\ensuremath{\boldsymbol{\xi}^{s1}}}
\newcommand{\htz}{\ensuremath{\boldsymbol{\xi}^{t0}}}
\newcommand{\hidl}{\ensuremath{\boldsymbol{\xi}^{l}}} % Common hidden layer l vector
\newcommand{\hidlm}{\ensuremath{\boldsymbol{\xi}^{l-1}}} % Common hidden layer l-1 vector
\newcommand{\hidz}{\ensuremath{\boldsymbol{\xi}^{0}}}% Common hidden layer 0 vector
\newcommand{\hidone}{\ensuremath{\boldsymbol{\xi}^{1}}}% Common hidden layer 1vector
\newcommand{\hidL}{\ensuremath{\boldsymbol{\xi}^{L}}}% Common hidden layer 1vector
\newcommand{\hsL}{\ensuremath{\boldsymbol{\xi}^{sL}}} % Source final layer L
\newcommand{\htL}{\ensuremath{\boldsymbol{\xi}^{tL}}}% Target final layer L
\newcommand{\hsLm}{\ensuremath{\boldsymbol{\xi}^{s(L-1)}}} % Source hidden layer L-1
\newcommand{\htLm}{\ensuremath{\boldsymbol{\xi}^{t(L-1)}}} %Target hidden layer L-1
\newcommand{\hidLm}{\ensuremath{\boldsymbol{\xi}^{L-1}}} % Common hidden layer L-1 vector

% Network parameters
\newcommand{\Wsl}{\ensuremath{\mathbf{W}^{sl}}} % Source weight layer l
\newcommand{\Wtl}{\ensuremath{\mathbf{W}^{tl}}} %Target weight layer l
\newcommand{\Wl}{\ensuremath{\mathbf{W}^{l}}} %General weight layer l
\newcommand{\Wslm}{\ensuremath{\mathbf{W}^{s(l-1)}}} % Source weight layer l-1
\newcommand{\Wtlm}{\ensuremath{\mathbf{W}^{t(l-1)}}} %Target weight layer l-1
\newcommand{\WsL}{\ensuremath{\mathbf{W}^{sL}}} % Source weight layer L
\newcommand{\WtL}{\ensuremath{\mathbf{W}^{tL}}} % Target weight layer L
\newcommand{\WL}{\ensuremath{\mathbf{W}^{L}}} % Common weight layer L
\newcommand{\WsLm}{\ensuremath{\mathbf{W}^{s(L-1)}}} % Source weight layer L-1
\newcommand{\WtLm}{\ensuremath{\mathbf{W}^{t(L-1)}}} %Target weight layer L-1
\newcommand{\Wone}{\ensuremath{\mathbf{W}^{1}}} % Common weight layer 1
\newcommand{\Wsone}{\ensuremath{\mathbf{W}^{s1}}} % Source weight layer 1

% Domain-adversarial
\newcommand{\WKdan}{\ensuremath{\mathbf{W}^{K}_\ddan}} % Common weight layer K
\newcommand{\Wldan}{\ensuremath{\mathbf{W}^{l}_\ddan}}
\newcommand{\Wlmdan}{\ensuremath{\mathbf{W}^{l-1}_\ddan}}
\newcommand{\Wonedan}{\ensuremath{\mathbf{W}^{1}_\ddan}} % Common weight layer 1
\newcommand{\bKdan}{\ensuremath{\mathbf{b}^{K}_\ddan}} % Common weight layer K
\newcommand{\bldan}{\ensuremath{\mathbf{b}^{l}_\ddan}}
\newcommand{\blmdan}{\ensuremath{\mathbf{b}^{l-1}_\ddan}}
\newcommand{\bonedan}{\ensuremath{\mathbf{b}^{1}_\ddan}} % Common weight layer 1


\newcommand{\Thetasl}{\ensuremath{\boldsymbol{\Theta}^{sl}}} 
\newcommand{\Thetatl}{\ensuremath{\boldsymbol{\Theta}^{tl}}} 
\newcommand{\Thetas}{\ensuremath{\boldsymbol{\Theta}^{s}}} 
\newcommand{\Thetat}{\ensuremath{\boldsymbol{\Theta}^{t}}} 
\newcommand{\ThetaCom}{\ensuremath{\boldsymbol{\Theta}}} 
\newcommand{\ThetaSetl}{\ensuremath{\mathbf{\boldsymbol{\Theta}}^l}} 
\newcommand{\Thetal}{\ensuremath{\boldsymbol{\Theta}^{l}}} 
\newcommand{\Phis}{\ensuremath{\mathbf{\Phi}^{s}}} 
\newcommand{\Phit}{\ensuremath{\mathbf{\Phi}^{t}}} 
\newcommand{\PhiCom}{\ensuremath{\mathbf{\Phi}}} 
\newcommand{\mapFs}{\ensuremath{\mathcal{M}_{\Fs}}} 
\newcommand{\mapGs}{\ensuremath{\mathcal{M}_{\Gs}}} 

\newcommand{\bsl}{\ensuremath{\mathbf{b}^{sl}}} % Source bias layer l
\newcommand{\btl}{\ensuremath{\mathbf{b}^{tl}}} %Target bias layer l
\newcommand{\bl}{\ensuremath{\mathbf{b}^{l}}} %General bias layer l
\newcommand{\bslm}{\ensuremath{\mathbf{b}^{s(l-1)}}} % Source bias layer l-1
\newcommand{\btlm}{\ensuremath{\mathbf{b}^{t(l-1)}}} %Target bias layer l-1
\newcommand{\bsL}{\ensuremath{\mathbf{b}^{sL}}} % Source bias layer l
\newcommand{\btL}{\ensuremath{\mathbf{b}^{tL}}} % Source bias layer l
\newcommand{\bL}{\ensuremath{\mathbf{b}^{L}}} %Common bias layer l
\newcommand{\bsLm}{\ensuremath{\mathbf{b}^{s(L-1)}}} % Source bias layer L-1
\newcommand{\btLm}{\ensuremath{\mathbf{b}^{t(L-1)}}} %Target bias layer L-1
\newcommand{\bsone}{\ensuremath{\mathbf{b}^{s1}}} % Source bias layer 1
\newcommand{\bone}{\ensuremath{\mathbf{b}^{1}}} % Common bias layer 1


% Topology
\newcommand{\numL}{\ensuremath{L}} %Number of layers
\newcommand{\numLdan}{\ensuremath{K}} %Number of layers
\newcommand{\actl}{\ensuremath{\eta^l}} %Activation function layer l
\newcommand{\act}{\ensuremath{\eta}} %Activation function 
\newcommand{\actone}{\ensuremath{\eta^1}} %Activation function layer 1
\newcommand{\lay}{\ensuremath{l}} % Layer index


% Dimensions
\newcommand{\dl}{\ensuremath{{d_l}}}
\newcommand{\dlm}{\ensuremath{{d_{l-1}}}}
\newcommand{\dz}{\ensuremath{{d_0}}}
\newcommand{\done}{\ensuremath{{d_1}}}
\newcommand{\dk}{\ensuremath{{d_k}}}
\newcommand{\dkm}{\ensuremath{{d_{k-1}}}}
\newcommand{\di}{\ensuremath{{d_i}}}
\newcommand{\dimm}{\ensuremath{{d_{i-1}}}}
\newcommand{\dkplone}{\ensuremath{{d_{k+1}}}}
\newcommand{\dL}{\ensuremath{{d_L}}}
\newcommand{\dLm}{\ensuremath{{d_{L-1}}}}
\newcommand{\dlmax}{\ensuremath{{d_{\max}}}}
\newcommand{\Rdl}{\ensuremath{\mathbb{R}^{d_l}}}
\newcommand{\dcom}{\ensuremath{{d}}}

\newcommand{\dlddan}{\ensuremath{{d^{\ddan}_l}}}
\newcommand{\dlmddan}{\ensuremath{{d^{\ddan}_{l-1}}}}
\newcommand{\dKddan}{\ensuremath{{d^{\ddan}_K}}}
% Bounds
\newcommand{\Bnet}{\ensuremath{{A_\Theta}}} % Bound on network parameters
\newcommand{\Leta}{\ensuremath{{L_\eta}}} 
\newcommand{\Binp}{\ensuremath{{A_x}}}  % Bound on xs norm
\newcommand{\Beta}{\ensuremath{{C_\eta}}} % Activation function bound
\newcommand{\Bddan}{\ensuremath{{C_\Dspace}}} % Domain discriminator bound
\newcommand{\Bddansq}{\ensuremath{{C^2_\Dspace}}} % Domain discriminator bound

\newcommand{\Bopeta}{\ensuremath{{A_\eta}}} % Activation function operator bound
\newcommand{\Bdiml}{\ensuremath{{R_l}}} % Dimension-dependent bound for layer l
\newcommand{\Bdimlm}{\ensuremath{{R_{l-1}}}} % Dimension-dependent bound for layer l
\newcommand{\BdimL}{\ensuremath{{R_L}}} % Dimension-dependent bound for layer L
\newcommand{\BdimLm}{\ensuremath{{R_{L-1}}}} 
\newcommand{\Bdimz}{\ensuremath{{R_0}}}
\newcommand{\Bdimone}{\ensuremath{{R_1}}}
\newcommand{\Bdimi}{\ensuremath{{R_i}}} 
\newcommand{\Bdimim}{\ensuremath{{R_{i-1}}}} 
\newcommand{\BQdiml}{\ensuremath{{Q_l}}} 
\newcommand{\BQdimone}{\ensuremath{{Q_1}}} 
\newcommand{\BQdimL}{\ensuremath{{Q_L}}} 
\newcommand{\BQdimLm}{\ensuremath{{Q_{L-1}}}} 
\newcommand{\BQ}{\ensuremath{{Q}}} 


% RKHS
\newcommand{\phil}{\ensuremath{\phi^l}} % Feature map layer l
\newcommand{\kr}{\ensuremath{k}} % Kernel 
\newcommand{\krl}{\ensuremath{k^l}} % Kernel layer l
\newcommand{\Xl}{\ensuremath{\mathcal{X}^l}}

% Domain-adversarial networks
\newcommand{\ddan}{\ensuremath{\Delta}} % Domain discriminator hypothesis 
\newcommand{\dom}{\ensuremath{\mathcal{\D}}} % Domain symbol
\newcommand{\ydoms}{\ensuremath{l^s}} % Source domain label
\newcommand{\ydomt}{\ensuremath{l^t}} % Target domain label
\newcommand{\actldan}{\ensuremath{\eta^l_\ddan}} %Activation function layer l
\newcommand{\hidldan}{\ensuremath{\boldsymbol{\xi}^{l}_\ddan}}
\newcommand{\hidlmdan}{\ensuremath{\boldsymbol{\xi}^{l-1}_\ddan}}
\newcommand{\hidzdan}{\ensuremath{\boldsymbol{\xi}^{0}_\ddan}}
\newcommand{\hidKdan}{\ensuremath{\boldsymbol{\xi}^{K}_\ddan}}
\newcommand{\actKdan}{\ensuremath{\eta^\numLdan_\ddan}} %Activation function layer l




%%% Local Variables: 
%%% mode:latex
%%% TeX-master: "main"
%%% End: 
